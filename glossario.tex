
%**************************************************************
% Acronimi
%**************************************************************
\renewcommand{\acronymname}{Acronimi e abbreviazioni}

\newacronym[description={\glslink{apig}{Application Program Interface}}]
    {api}{API}{Application Program Interface}

\newacronym[description={\glslink{umlg}{Unified Modeling Language}}]
    {uml}{UML}{Unified Modeling Language}

%**************************************************************
% Glossario
%**************************************************************
%\renewcommand{\glossaryname}{Glossario}

\newglossaryentry{url}
{
    name=\glslink{url}{URL},
    text=Uniform Resource Locator,
    sort=url,
    description={in informatica con il termine \emph{Uniform Resource Locator URL} si indica una sequenza di caratteri che identifica univocamente l'indirizzo di una risorsa su una rete di computer, come ad esempio un documento, un'immagine, un video, tipicamente presente su un host server e resa accessibile a un client. È perlopiù utilizzato per indicare risorse web (http), risorse recuperabili tramite protocolli di trasferimento file (ftp), condivisioni remote (smb) o accessi a sistemi esterni (ssh). La risoluzione dell'URL in indirizzo IP, necessario per l'instradamento con il protocollo IP avviene tramite DNS. }
}

\newglossaryentry{umlg}
{
    name=\glslink{uml}{UML},
    text=UML,
    sort=uml,
    description={in ingegneria del software \emph{UML, Unified Modeling Language} (ing. linguaggio di modellazione unificato) è un linguaggio di modellazione e specifica basato sul paradigma object-oriented. L'\emph{UML} svolge un'importantissima funzione di ``lingua franca'' nella comunità della progettazione e programmazione a oggetti. Gran parte della letteratura di settore usa tale linguaggio per descrivere soluzioni analitiche e progettuali in modo sintetico e comprensibile a un vasto pubblico}
}
