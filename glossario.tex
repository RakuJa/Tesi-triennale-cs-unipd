%**************************************************************
% Glossario
%**************************************************************
%\renewcommand{\glossaryname}{Glossario}

\newglossaryentry{Tor}
{
    name=\glslink{Tor}{TOR},
    text=Tor,
    sort=tor,
    description={In informatica \emph{Tor, The Onion Router} è un software libero, rilasciato su licenza BSD, con un'interfaccia di gestione disponibile (Vidalia), che permette una comunicazione anonima per Internet basata sulla seconda generazione del protocollo di rete di onion routing: tramite il suo utilizzo è molto più difficile tracciare l'attività Internet dell'utente essendo finalizzato a proteggere la privacy degli utenti, la loro libertà e la possibilità di condurre delle comunicazioni confidenziali senza che vengano monitorate o intercettate.}
}

\newglossaryentry{I2P}
{
    name=\glslink{I2P}{I2P},
    text=I2P,
    sort=I2P,
    description={In informatica \emph{I2P, Invisible Internet Project} è un software per la realizzazione di una rete anonima (implementata come un mix di reti) che permette una comunicazione peer-to-peer e resistente alle censure. Le connessioni anonime vengono realizzate criptando il traffico di rete dell'utente (usando end-to-end encryption) ed inoltrando il traffico in una rete mantenuta da volontari in tutto il mondo. Data l'alto numero di possibili percorsi nei quali il traffico può transitare, la possibilità che un terzo possa osservare l'intera connessione è remota. Il software che implementa questo strato è chiamato "router I2P" ed un computer che opera su I2P è chiamato "nodo I2P". I2P è gratuito ed open source, pubblicato con una moltitudine di licenze.}
}

\newglossaryentry{VCS}
{
    name=\glslink{VCS}{VCS},
    text=VCS,
    sort=VCS,
    description={Il controllo versione, in informatica, è la gestione di versioni multiple di un insieme di informazioni: gli strumenti software per il controllo versione sono ritenuti molto spesso necessari per la maggior parte dei progetti di sviluppo software o documentali gestiti da un team collaborativo di sviluppo o redazione}
}

\newglossaryentry{broker}
{
    name=\glslink{broker}{BROKER},
    text=broker di messaggistica,
    sort=broker,
    description={Un broker di messaggi è un programma intermedio
che traduce un messaggio dal protocollo di messaggistica formale del mittente al
protocollo di messaggistica formale del ricevitore. I broker di messaggi sono elementi
di telecomunicazione o reti di computer in cui le applicazioni software comunicano
scambiando messaggi definiti in modo formale.}
}

\newglossaryentry{ttl}
{
    name=\glslink{ttl}{TTL},
    text=time to live,
    sort=broker,
    description={\emph{TTL, Time to live} è un meccanismo tramite il quale si limita la longevità di dati in un computer o in una rete. TTL può essere implementato come un contatore o come una marca temporale allegata nel dato. Una volta che il contatore o la marca temporale sono scaduti, il dato viene scartato o rivalutato. In informatica, il TTL è comunemente usato per migliorare le prestazioni e la gestione di cache di dati.}
}


