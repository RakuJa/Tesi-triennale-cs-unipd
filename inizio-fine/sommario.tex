% !TEX encoding = UTF-8
% !TEX TS-program = pdflatex
% !TEX root = ../tesi.tex

%**************************************************************
% Abstract
%**************************************************************
\cleardoublepage
\phantomsection
\pdfbookmark{Abstract}{Abstract}
\begingroup
\let\clearpage\relax
\let\cleardoublepage\relax
\let\cleardoublepage\relax

\chapter*{Abstract}

Il presente documento descrive il lavoro svolto durante il periodo di stage, della durata di trecentoventi ore, dal laureando \myName{} presso l'azienda \myCompany{} \myCompanyRag{}.
L'obiettivo dello stage era la creazione di un web crawler intelligente da integrare nell’attuale piattaforma di cyber intelligence proprietaria, per aumentare le informazioni raccolte dalle varie sorgenti già presenti. Questo si è tradotto nell'implementazione di un nuovo modulo composto da due componenti: uno per l'inserimento periodico di url di partenza denominato "master" ed uno che effettua l'analisi degli url denominato "worker". \newline{}
Nella prima fase del progetto ho studiato la struttura della piattaforma proprietaria, il funzionamento del dark web ed il funzionamento dei più famosi web scraper. Infine vi è stata una fase di sviluppo seguita da continui miglioramenti all'applicativo, con lo scopo di aggiungere funzionalità ed ottimizzare il più possibile le prestazioni.

%\vfill
%
%\selectlanguage{english}
%\pdfbookmark{Abstract}{Abstract}
%\chapter*{Abstract}
%
%\selectlanguage{italian}

\endgroup			

\vfill

