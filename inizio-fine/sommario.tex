% !TEX encoding = UTF-8
% !TEX TS-program = pdflatex
% !TEX root = ../tesi.tex

%**************************************************************
% Abstract
%**************************************************************
\cleardoublepage
\phantomsection
\pdfbookmark{Abstract}{Abstract}
\begingroup
\let\clearpage\relax
\let\cleardoublepage\relax
\let\cleardoublepage\relax

\chapter*{Abstract}

Il presente documento descrive il lavoro svolto durante il periodo di stage, della durata di trecentoventi ore (320), dal laureando \myName{} presso l'azienda \myCompany{} \myCompanyRag{}.
L'obbiettivo principale dello stage era la creazione di un web crawler intelligente da integrare nell’attuale piattaforma
di cyber intelligence proprietaria, con l’obiettivo di aumentare le informazioni raccolte dalle varie
sorgenti già presenti. Questo si è tradotto nell'implementazione di due nuovi moduli nella piattaforma: uno per l'inserimento periodico di url di partenza ed uno per la ricerca tramite questi url. \newline{}
In una prima fase ho quindi appreso il funzionamento della piattaforma proprietaria e del funzionamento della rete Tor. Successivamente ho potuto comprendere il funzionamento dei principali web scraper 

%\vfill
%
%\selectlanguage{english}
%\pdfbookmark{Abstract}{Abstract}
%\chapter*{Abstract}
%
%\selectlanguage{italian}

\endgroup			

\vfill

