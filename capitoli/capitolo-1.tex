% !TEX encoding = UTF-8
% !TEX TS-program = pdflatex
% !TEX root = ../tesi.tex

%**************************************************************
\chapter{Introduzione}
\label{cap:introduzione}
%**************************************************************

\intro{Il seguente capitolo contiene un’introduzione all’azienda, al progetto di stage e all’organiz-
zazione del testo.}\newline{}

%**************************************************************
\section{L'azienda}

Fondata nel 2001, Yarix S.r.l. è la società a capo della divisione Digital Security di Var Group e una delle aziende italiane più riconosciute, innovative e autorevoli nel comparto della sicurezza informatica: da 20 anni fornisce servizi e soluzioni di cyber security, business continuity e disaster recovery a industrie, enti governativi e militari, aziende del comparto sanitario e università. Yarix è oggi tra i più importanti player sul territorio nazionale.\newline{}
Yarix dispone di un laboratorio di ricerca e sviluppo a Tel Aviv, dove un gruppo di ingegneri informatici progetta e realizza le soluzioni e i sistemi di sicurezza più all’avanguardia, e di uno dei più importanti e tecnologicamente evoluti Cognitive SOC (Security Operation Center) per il monitoraggio delle reti aziendali. Questo bunker informatico è attivo tutti i giorni, ventiquattr’ore su ventiquattro, e dispone di tutte le misure di sicurezza fisica e biometrica di ultima generazione. Attraverso l’utilizzo delle più recenti e innovative soluzioni tecnologiche nel campo della security, il servizio permette di intercettare proattivamente e bloccare qualsiasi segnale di un tentativo di attacco.\newline{}
Il 2014 ha visto l’avvio di una strategia di crescita finalizzata alla creazione di un polo di eccellenza per la gestione globale della sicurezza delle imprese. Attraverso un processo di acquisizioni che ha portato all’integrazione di realtà col maggiore potenziale e le competenze più evolute in Italia, Var Group e Yarix offrono alle aziende italiane, che affrontano le sfide dell’innovazione tecnologica e della trasformazione digitale, un nuovo livello di protezione. La sicurezza oggi non può più essere né solo fisica, né solo informatica, ma richiede una visione integrata e d’insieme. Yarix offre un approccio globale ed olistico alla Security, attraverso il monitoraggio costante e un’analisi obiettiva di tutti i contesti, per ottenere risposte adeguate a prevenire i rischi per le aziende.

\begin{figure}[!h] 
    \centering 
    \includegraphics[width=0.3\columnwidth]{chapter1-introduction/logo-yarix.png} 
    \caption{Logo Yarix S.r.l.}
\end{figure}

%**************************************************************
\section{Il progetto}

Nel contesto di una piattaforma proprietaria di raccolta e classificazione di informazioni di Cyber Threat Intelligence, il progetto prevede lo sviluppo di un sistema di crawling e web-scraping, scalabile e performante, con lo scopo di automatizzare la fase di raccolta informazioni e dati su rete non indicizzata (Tor). Partendo da una fase di analisi atta ad identificare metodologie e migliori approcci, si è sviluppato tale sistema utilizzando logiche e algoritmi che possano indirizzare la raccolta sulle informazioni ritenute più rilevanti e significative rispetto ad un contesto funzionale definito.\newline{}
Il progetto è stato integrato nell'attuale piattaforma, con l'obiettivo di estenderne le funzionalità ed aumentare le informazioni raccolte dalle varie sorgenti.\newline{} Gli obbiettivi principali del tirocinio erano: \newline{}
\begin{itemize}
	\item Studio e analisi dell'applicativo utilizzato dall’azienda;
	\item Analisi e progettazione del nuovo modulo;
	\item Implementazione del modulo di crawling e ottimizzazione;
	\item Verifica e collaudo del prodotto ultimato.
\end{itemize}
%**************************************************************
\section{Pianificazione}
Lo stage è stato svolto dal 28/06/2021 al 20/08/2021 per un totale di 320 ore. La pianificazione del lavoro è stata suddivisa in 8 settimane in modalità mista, con tre giorni a settimana in remoto e due in presenza. La prima settimana è stata dedicata alla formazione sulle tecnologie e sull'architettura esistente; ho studiato da documenti forniti dal tutor aziendale su: struttura della piattaforma sviluppata, rete Tor, Celery, RabbitMQ, Selenium ed infine requests e beautifulsoap.\newline{}
La seconda settimana è stata dedicata allo sviluppo di snippet di codice e alla progettazione del modulo. \newline{}
La terza e quarta settimana sono state dedicate allo sviluppo della base di codice, alla creazione di un ambiente di test e alla creazione di una libreria che permettesse di visualizzare i dati raccolti durante i test. Durante queste due settimane sono inoltre stati risolti problemi di memory leak scaturiti dall'utilizzo della libreria Celery.
La quinta settimana è stata dedicata alla risoluzione di bug, miglioramento delle performance e l'implementazione di Redis.
La sesta settimana è stata dedicata alla messa in produzione del programma e allo studio e utilizzo di Selenium.
La settima settimana è stata dedicata ad un refactor del codice e all'estensione delle funzionalità di webscraping su rete I2P.
Infine, l'ultima settimana è stata dedicata alla risoluzione di bug critici, implementazione di s3, aggiunta di funzionalità di screenshot e di diverse altre funzionalità facoltative.


\section{Contributo}
Questo stage ha contribuito a formarmi in un ambito a me completamente nuovo come quello del dark web e dei web crawler, permettendomi di fare esperienza nel lavoro su strutture complesse, come quella della piattaforma di Cyber Threat Intelligence proprietaria. Inoltre mi ha permesso di imparare a conoscere e ad utilizzare nuove tecnologie e di consolidare le tecnologie apprese nel corso dei tre anni di università. Il progetto, in particolare, mi ha permesso di creare un nuovo modulo che coprisse una area prima non ricercata ed inoltre ha potuto aiutarmi nel rafforzare le mie conoscenze in aree più tecniche nell'ambito dei sistemi come l'utilizzo di sistemi unix, la configurazione di macchine e la configurazione di reti. A livello aziendale, invece, questo stage ha permesso di dotare la piattaforma aziendale dei mezzi necessari per poter effettuare scraping su buona parte del dark e clear web indipendentemente dalla struttura del sito. Ciò significa che l’azienda possiede così uno strumento personalizzato ed estendibile per soddisfare le proprie esigenze. \newline{}
Il progetto è stato supervisionato dal tutor aziendale Matteo Neri.

%**************************************************************
\section{Organizzazione del testo}

\begin{description}
    \item[{\hyperref[cap:analisi-del-problema]{Il secondo capitolo}}] riguarda l'analisi dettagliata del problema e del prodotto, e comprende la lista dei requisiti individuati;
    
    \item[{\hyperref[cap:strumenti-e-tecnologie]{Il terzo capitolo}}] specifica quali strumenti e tecnologie sono stati analizzati e utilizzati;
    
    \item[{\hyperref[cap:progettazione-codifica]{Il quarto capitolo}}] approfondisce la progettazione e lo sviluppo del modulo;
    
    \item[{\hyperref[cap:verifica-validazione]{Il quinto capitolo}}] approfondisce le metodologie e le attività di verifica e validazione del modulo;
    
    \item[{\hyperref[cap:conclusioni]{Il sesto capitolo}}] riassume la valutazione personale del progetto e le conoscenze acquisite.
\end{description}

%**************************************************************
\section{Convenzioni tipografiche}

Riguardo la stesura del testo, relativamente al documento sono state adottate le seguenti convenzioni tipografiche:
\begin{itemize}
	\item gli acronimi, le abbreviazioni e i termini ambigui o di uso non comune menzionati vengono definiti nel glossario, situato alla fine del presente documento;
	\item per tutte le occorrenze dei termini riportati nel glossario viene utilizzata la seguente nomenclatura: \emph{parola}\glsfirstoccur;
	\item i termini riguardanti nomi di file, nomi di classi od oggetti sono segnalati con il carattere \texttt{termine}
	\item i termini in lingua straniera o facenti parti del gergo tecnico sono evidenziati con il carattere \emph{corsivo}.
\end{itemize}