% !TEX encoding = UTF-8
% !TEX TS-program = pdflatex
% !TEX root = ../tesi.tex

%**************************************************************
\chapter{Introduzione}
\label{cap:introduzione}
%**************************************************************

\intro{Il seguente capitolo contiene un’introduzione all’azienda, al progetto di stage e all’organiz-
zazione del testo.}\newline{}

%**************************************************************
\section{L'azienda}

Fondata nel 2001, Yarix S.r.l. è la società a capo della divisione Digital Security di Var Group e una delle aziende italiane più riconosciute, innovative e autorevoli nel comparto della sicurezza informatica: da 20 anni fornisce servizi e soluzioni di cyber security, business continuity e disaster recovery a industrie, enti governativi e militari, aziende del comparto sanitario e università. Yarix ha sede principale a Montebelluna ed è ad oggi tra i più importanti player sul territorio nazionale.\newline{}
Attraverso un processo di acquisizioni che ha portato all’integrazione di realtà col maggiore potenziale e le competenze più evolute in Italia, Var Group e Yarix offrono alle aziende italiane, che affrontano le sfide dell’innovazione tecnologica e della trasformazione digitale, un nuovo livello di protezione. Yarix offre un approccio globale ed olistico alla Security, attraverso il monitoraggio costante e un’analisi obiettiva di tutti i contesti, per ottenere risposte adeguate a prevenire i rischi per le aziende.

\begin{figure}[!h] 
    \centering 
    \includegraphics[width=0.3\columnwidth]{chapter1-introduction/logo-yarix.png} 
    \caption{Logo Yarix S.r.l.}
\end{figure}

%**************************************************************
\section{Il progetto}

Nel contesto di una piattaforma proprietaria di raccolta e classificazione di informazioni di \emph{Cyber Threat Intelligence}, il progetto prevedeva lo sviluppo di un sistema di \emph{crawling} e \emph{web-scraping}, scalabile e performante, con lo scopo di automatizzare la fase di raccolta informazioni e dati su rete non indicizzata. Partendo da una fase di analisi atta ad identificare metodologie e migliori approcci, si è sviluppato tale sistema utilizzando logiche e algoritmi che possano indirizzare la raccolta sulle informazioni ritenute più rilevanti e significative rispetto ad un contesto funzionale definito.\newline{}
Il progetto è stato integrato nella preesistente piattaforma, con l'obiettivo di estenderne le funzionalità ed aumentare le informazioni raccolte dalle varie sorgenti.\newline{} Gli obiettivi principali del tirocinio erano: \newline{}
\begin{itemize}
	\item Studio e analisi dell'applicativo utilizzato dall’azienda;
	\item Analisi e progettazione del nuovo modulo;
	\item Implementazione del modulo di crawling e ottimizzazione;
	\item Verifica e collaudo del prodotto ultimato.
\end{itemize}
%**************************************************************
\section{Pianificazione}
Lo stage è stato svolto dal 28/06/2021 al 20/08/2021 per un totale di 320 ore. La pianificazione del lavoro è stata suddivisa in 8 settimane in modalità mista, con tre giorni a settimana in remoto e due in presenza. La prima settimana è stata dedicata alla formazione sulle tecnologie e sull'architettura esistente; ho studiato documenti forniti dal tutor aziendale riguardanti: struttura della piattaforma sviluppata, rete \gls{Tor}, \emph{Celery}, \emph{RabbitMQ}, \emph{Selenium} ed infine \emph{requests} e \emph{beautifulsoap}.\newline{}
La seconda settimana è stata dedicata allo sviluppo di \emph{snippet} di codice e alla progettazione del modulo. \newline{}
La terza e quarta settimana sono state dedicate allo sviluppo della base di codice, alla creazione di un ambiente di test e alla creazione di una libreria che permettesse di visualizzare i dati raccolti durante i test. Durante queste due settimane sono inoltre stati risolti problemi riguardanti l'eccessivo consumo di memoria scaturiti dall'utilizzo della libreria \emph{Celery}.
La quinta settimana è stata dedicata alla risoluzione di \emph{bug}, miglioramento delle prestazioni e l'implementazione di \emph{Redis}.
La sesta settimana è stata dedicata alla messa in produzione del programma e allo studio e utilizzo di \emph{Selenium}.
La settima settimana è stata dedicata ad un \emph{refactor} del codice e all'estensione delle funzionalità di webscraping su rete \gls{I2P}.
Infine, l'ultima settimana è stata dedicata alla risoluzione di \emph{bug} critici, implementazione di \emph{Amazon S3}, aggiunta di funzionalità di screenshot e di diverse altre funzionalità facoltative.
%**************************************************************
\section{Organizzazione del testo}

\begin{description}
    \item[{\hyperref[cap:analisi-del-problema]{Il secondo capitolo}}] riguarda l'analisi dettagliata del problema e del prodotto, e comprende la lista dei requisiti individuati;
    
    \item[{\hyperref[cap:strumenti-e-tecnologie]{Il terzo capitolo}}] specifica quali strumenti e tecnologie sono stati analizzati e utilizzati;
    
    \item[{\hyperref[cap:progettazione-codifica]{Il quarto capitolo}}] approfondisce la progettazione e lo sviluppo del modulo;
    
    \item[{\hyperref[cap:verifica-validazione]{Il quinto capitolo}}] approfondisce le metodologie e le attività di verifica e validazione del modulo;
    
    \item[{\hyperref[cap:conclusioni]{Il sesto capitolo}}] riassume la valutazione personale del progetto e le conoscenze acquisite.
\end{description}

%**************************************************************
\section{Convenzioni tipografiche}

Riguardo la stesura del testo, relativamente al documento sono state adottate le seguenti convenzioni tipografiche:
\begin{itemize}
	\item gli acronimi, le abbreviazioni e i termini ambigui o di uso non comune menzionati vengono definiti nel glossario, situato alla fine del presente documento;
	\item per tutte le occorrenze dei termini riportati nel glossario viene utilizzata la seguente sintassi: \gls{ttl};
	\item i termini in lingua straniera o facenti parti del gergo tecnico sono evidenziati con il carattere \emph{corsivo}.
\end{itemize}