% !TEX encoding = UTF-8
% !TEX TS-program = pdflatex
% !TEX root = ../tesi.tex

%**************************************************************
\chapter{Strumenti e tecnologie}
\label{cap:strumenti-e-tecnologie}
%**************************************************************

\intro{Analisi e descrizione delle tecnologie utilizzate}\\

%**************************************************************
\section{Strumenti utilizzati}

\subsection{Ambiente di sviluppo}
Durante le 8 settimane di stage, mi è stata assegnata una macchina con sistema operativo Windows 10 64 bit per svolgere il tirocinio universitario. Mi è stato richiesto di utilizzare VirtualBox per poter creare l'ambiente di sviluppo e l'ambiente di test. \emph{VirtualBox} è un software gratuito e open source per l'esecuzione di macchine virtuali per architettura x86 e 64bit che supporta Windows, GNU/Linux e macOS come sistemi operativi host, ed è in grado di eseguire Windows, GNU/Linux, OS/2 Warp, BSD come ad esempio OpenBSD, FreeBSD e infine Solaris e OpenSolaris come sistemi operativi guest.\newline{}
Nella macchina virtuale in seguito mi è stato consigliato di utilizzare \emph{PyCharm} come strumento per lo sviluppo del prodotto. \emph{PyCharm} è un \emph{IDE} integrato multi-piattaforma per \emph{Python}; è un software distribuito sotto i termini della Apache License nella sua versione "Community".
\begin{figure}[!h]
    \begin{minipage}{.5\textwidth} 
        \centering 
        \includegraphics[width=.4\linewidth]{chapter3-tech/tools/develop/logo-virtualbox.png} 
        \caption{Logo Virtualbox} 
        \label{fig:virtualbox} 
    \end{minipage}% 
    \begin{minipage}{.5\textwidth} 
        \centering 
        \includegraphics[width=.4\linewidth]{chapter3-tech/tools/develop/logo-pycharm.png} 
        \caption{Logo PyCharm} 
        \label{fig:pycharm} 
    \end{minipage}  
\end{figure}

\noindent
Nelle macchine virtuali, sono stati da me configurati due ambienti con due sistemi operativi diversi. L'ambiente di sviluppo è stato configurato con sistema operativo Debian 10.10 (Buster) mentre l'ambiente di test è stato configurato per simulare il più possibile l'ambiente in produzione, utilizzando quindi come sistema operativo CentOS 7.
\begin{figure}[!h]
    \begin{minipage}{.5\textwidth} 
        \centering 
        \includegraphics[width=.4\linewidth]{chapter3-tech/tools/develop/logo-debian.png} 
        \caption{Logo Debian} 
        \label{fig:virtualbox} 
    \end{minipage}% 
    \begin{minipage}{.5\textwidth} 
        \centering 
        \includegraphics[width=.4\linewidth]{chapter3-tech/tools/develop/logo-centos.png} 
        \caption{Logo CentOS} 
        \label{fig:pycharm} 
    \end{minipage}  
\end{figure}
\subsection{Strumenti di versionamento}

L’azienda, per tutti i suoi prodotti, adotta strumenti per il controllo di versione, nello specifico \emph{Git}. \emph{Git} è un software di controllo versione distribuito utilizzabile da interfaccia a riga di comando, creato da Linus Torvalds nel 2005. È stato scelto, rispetto ad altri strumenti di \gls{VCS}, per il supporto di branching e merging e perché comprende strumenti specifici per visualizzare e navigare una cronologia di sviluppo non lineare. Infine, come servizio di hosting web-based, è stato utilizzato \emph{GitLab}.

\begin{figure}[!h]
    \begin{minipage}{.5\textwidth} 
        \centering 
        \includegraphics[width=.4\linewidth]{chapter3-tech/tools/versioning/logo-git.png} 
        \caption{Logo Git} 
        \label{fig:git} 
    \end{minipage}% 
    \begin{minipage}{.5\textwidth} 
        \centering 
        \includegraphics[width=.4\linewidth]{chapter3-tech/tools/versioning/logo-gitlab.png} 
        \caption{Logo GitLab} 
        \label{fig:gitlab} 
    \end{minipage}  
\end{figure}

\subsection{Strumenti aziendali}

Per quanto riguarda la comunicazione con il tutor aziendale, sono stati utilizzati \emph{Microsoft Teams} e Whatsapp. Entrambi gli strumenti sono stati utilizzati per comunicare con il tutor ed il personale aziendale durante tutto il tirocinio universitario.

\begin{figure}[!h]
    \begin{minipage}{.5\textwidth} 
        \centering 
        \includegraphics[width=.4\linewidth]{chapter3-tech/tools/company/logo-teams.png} 
        \caption{Logo Microsoft Teams} 
        \label{fig:teams} 
    \end{minipage}% 
    \begin{minipage}{.5\textwidth} 
        \centering 
        \includegraphics[width=.4\linewidth]{chapter3-tech/tools/company/logo-whatsapp.png} 
        \caption{Logo WhatsApp} 
        \label{fig:whatsapp} 
    \end{minipage}  
\end{figure}

\subsection{Altri strumenti}

Per avere un sistema di caching atto ad evitare di analizzare più volte lo stesso url in un breve lasso di tempo, si è utilizzato \emph{Redis}. \emph{Redis} è lo store di strutture dati chiave-valore in memoria più usato, è rapido ed open source. I suoi principali casi d'uso sono il caching, la gestione di sessioni, servizi pub/sub e graduatorie. Dispone di licenza BSD ed è un acronimo e sta per REmote DIctionary Server.
Per creare più istanze del programma e per immagazzinare in una struttura di dati flessibile gli url da analizzare, è stato utilizzato \emph{Celery}. \emph{Celery} è un gestore di code asincrono open source basato sul passaggio di messaggi distribuiti. Le unità di esecuzione, chiamate tasks, sono eseguite in concorrenza in uno o più nodi worker utilizzando il multiprocessing, eventlet o gevent.
\begin{figure}[!h]
    \begin{minipage}{.5\textwidth} 
        \centering 
        \includegraphics[width=.6\linewidth]{chapter3-tech/tools/others/logo-redis.png} 
        \caption{Redis} 
        \label{fig:redis} 
    \end{minipage}% 
    \begin{minipage}{.5\textwidth} 
        \centering 
        \includegraphics[width=.3\linewidth]{chapter3-tech/tools/others/logo-celery.png} 
        \caption{Logo Celery} 
        \label{fig:celery} 
    \end{minipage}% 
\end{figure}

\noindent
\emph{RabbitMQ} è un message-oriented middleware (detto anche \gls{broker}) che implementa il protocollo Advanced Message Queuing Protocol (AMQP). Esso è stato utilizzato per comunicare con \emph{Celery}. Il server \emph{RabbitMQ} è scritto in \emph{Erlang} e si basa sul framework Open Telecom Platform (OTP) per la gestione del \emph{clustering} e del \emph{failover}.
\begin{figure}[!h] 
    \centering 
    \includegraphics[width=0.4\columnwidth]{chapter3-tech/tools/others/logo-rabbitmq.png} 
    \caption{Logo RabbitMQ}
    \label{fig:rabbitmq} 
\end{figure}

\noindent
Per salvare lo stato dei siti interessanti per il modulo, si è deciso di immagazzinare il codice sorgente ed una istantanea del sito. Per il salvataggio si è utilizzato \emph{Amazon S3}. \emph{Amazon S3} (Simple Storage Service) è un servizio web di memorizzazione offerto da Amazon Web Services.\newline{} Per salvare e ricercare informazioni dettagliate sui siti interessanti per il modulo, viene utilizzato \emph{Elasticsearch}. \emph{Elasticsearch} è un server di ricerca basato su Lucene, con capacità Full Text e supporto ad architetture distribuite. Tutte le funzionalità sono nativamente esposte tramite interfaccia RESTful.
\begin{figure}[!h]
    \begin{minipage}{.5\textwidth} 
        \centering 
        \includegraphics[width=.5\linewidth]{chapter3-tech/tools/others/logo-amazons3.png} 
        \caption{Amazon S3} 
        \label{fig:amazons3} 
    \end{minipage}% 
    \begin{minipage}{.5\textwidth} 
        \centering 
        \includegraphics[width=.5\linewidth]{chapter3-tech/tools/others/logo-elasticsearch.png} 
        \caption{Logo Elasticsearch} 
        \label{fig:elasticsearch} 
    \end{minipage}% 
\end{figure} 

%**************************************************************
\section{Tecnologie utilizzate}

\subsection{Linguaggi}
Il principale linguaggio utilizzato è stato \emph{Python}, in versione 3.7.3. \emph{Python} è un linguaggio di programmazione ad alto livello, orientato agli oggetti, adatto, tra gli altri usi, a sviluppare applicazioni distribuite, \emph{scripting}, computazione numerica e \emph{system testing}. Sebbene \emph{Python} venga in genere considerato un linguaggio interpretato, in realtà il codice sorgente non viene convertito direttamente in linguaggio macchina. Infatti passa prima da una fase di pre-compilazione in \emph{bytecode}, che viene quasi sempre riutilizzato dopo la prima esecuzione del programma, evitando così di reinterpretare ogni volta il sorgente e migliorando le prestazioni. La scelta della versione è ricaduta su quella ritenuta più stabile e che non creasse conflitti con gli altri strumenti e tecnologie.\newline{}
Per la gestione e configurazione dei file di configurazione è stato utilizzato \emph{YAML}, un formato per la serializzazione di dati utilizzabile da esseri umani. Il nome definisce l'acronimo ricorsivo "YAML Ain't a Markup Language".
\begin{figure}[!h]
    \begin{minipage}{.5\textwidth} 
        \centering 
        \includegraphics[width=.4\linewidth]{chapter3-tech/languages/logo-python.png} 
        \caption{Python} 
        \label{fig:python} 
    \end{minipage}% 
    \begin{minipage}{.5\textwidth} 
        \centering 
        \includegraphics[width=.4\linewidth]{chapter3-tech/languages/logo-yaml.png} 
        \caption{Logo YAML} 
        \label{fig:yaml} 
    \end{minipage}% 
\end{figure}
\subsection{Framework}
\emph{Selenium} è un \emph{framework} per la gestione automatizzata dei browser. È stato utilizzato per simulare il comportamento di un utente reale all'interno di un browser evitando così i controlli \emph{antibot}. \emph{Selenium} è multipiattaforma, funziona su Windows, Linux e macOS, ed è un software open source rilasciato con la licenza Apache 2.0.
\begin{figure}[!h] 
    \centering 
    \includegraphics[width=0.4\columnwidth]{chapter3-tech/frameworks/logo-selenium.png} 
    \caption{Logo Selenium}
    label{fig:selenium} 
\end{figure}

%**************************************************************