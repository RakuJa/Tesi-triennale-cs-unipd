% !TEX encoding = UTF-8
% !TEX TS-program = pdflatex
% !TEX root = ../tesi.tex

%**************************************************************
\chapter{Strumenti e tecnologie}
\label{cap:strumenti-e-tecnologie}
%**************************************************************

\intro{Analisi e descrizione delle tecnologie utilizzate}\\

%**************************************************************
\section{Strumenti utilizzati}

\subsection{Ambiente di sviluppo}
Durante le 8 settimane di stage, mi è stata assegnata una macchina con sistema operativo Windows 10 64 bit per svolgere il tirocinio universitario. Mi è stato richiesto di utilizzare Virtual Box per poter creare l'ambiente di sviluppo con sistema operativo Debian 10.10 (Buster). \newline{} Virtual Box è un software gratuito e open source per l'esecuzione di macchine virtuali per architettura x86 e 64bit che supporta Windows, GNU/Linux e macOS come sistemi operativi host, ed è in grado di eseguire Windows, GNU/Linux, OS/2 Warp, BSD come ad esempio OpenBSD, FreeBSD e infine Solaris e OpenSolaris come sistemi operativi guest.\newline{}
Nella macchina virtuale in seguito mi è stato consigliato di utilizzare PyCharm come strumento per lo sviluppo del prodotto.\newline{}
PyCharm è un IDE integrato multi-piattaforma per Python; è un software distribuito sotto i termini della Apache License nella sua versione "Community".
\begin{figure}[!h]
    \begin{minipage}{.5\textwidth} 
        \centering 
        \includegraphics[width=.4\linewidth]{logo-virtualbox.png} 
        \caption{Logo Virtualbox} 
        \label{fig:virtualbox} 
    \end{minipage}% 
    \begin{minipage}{.5\textwidth} 
        \centering 
        \includegraphics[width=.4\linewidth]{logo-pycharm.png} 
        \caption{Logo PyCharm} 
        \label{fig:pycharm} 
    \end{minipage}  
\end{figure}

\subsection{Strumenti di versionamento}

L’azienda, per tutti i suoi prodotti, adotta strumenti per il controllo di versione, nello specifico Git. Git è un software di controllo versione distribuito utilizzabile da interfaccia a riga di comando, creato da Linus Torvalds nel 2005. È stato scelto rispetto ad altri strumenti di Version Control System (VCS) perché favorisce lo sviluppo non lineare, per il supporto di branching e merging e perché comprende strumenti specifici per visualizzare e navigare una cronologia di sviluppo non lineare. Infine, come servizio di hosting web-based, è stato utilizzato GitLab.

\begin{figure}[!h]
    \begin{minipage}{.5\textwidth} 
        \centering 
        \includegraphics[width=.4\linewidth]{logo-git.png} 
        \caption{Logo Git} 
        \label{fig:git} 
    \end{minipage}% 
    \begin{minipage}{.5\textwidth} 
        \centering 
        \includegraphics[width=.4\linewidth]{logo-gitlab.png} 
        \caption{Logo GitLab} 
        \label{fig:gitlab} 
    \end{minipage}  
\end{figure}

\subsection{Strumenti aziendali}

Per quanto riguarda la comunicazione con il tutor aziendale, è stato utilizzati Microsoft Teams e Whatsapp. Entrambi gli strumenti sono stati utililzzati per comunicare con il tutor e il personale aziendale durante tutto il tirocinio universitario.

\begin{figure}[!h]
    \begin{minipage}{.5\textwidth} 
        \centering 
        \includegraphics[width=.4\linewidth]{logo-teams.png} 
        \caption{Logo Microsoft Teams} 
        \label{fig:teams} 
    \end{minipage}% 
    \begin{minipage}{.5\textwidth} 
        \centering 
        \includegraphics[width=.4\linewidth]{logo-whatsapp.png} 
        \caption{Logo WhatsApp} 
        \label{fig:whatsapp} 
    \end{minipage}  
\end{figure}

\subsection{Altri strumenti}

\textit{Redis} è uno store di strutture dati chiave-valore in memoria rapido e open source. \textit{Redis} offre una serie di strutture dati in memoria molto versatili, che permettono di creare un'ampia gamma di applicazioni personalizzate. I principali casi d'uso sono il caching, la gestione di sessioni, servizi pub/sub e graduatorie. Si tratta dello store chiave-valore più usato. Dispone di licenza BSD, è scritto in C, il suo codice è ottimizzato e supporta diverse sintassi di sviluppo. \textit{Redis} è un acronimo e sta per REmote DIctionary Server.
\textit{Celery} è un gestore di code asincrono open source basato sul passaggio di messaggi distribuiti. Le unità di esecuzione, chiamate tasks, sono eseguite in concorrenza in uno o più nodi worker utilizzando il multiprocessing, eventlet o gevent. 


\begin{figure}[!h]
    \begin{minipage}{.5\textwidth} 
        \centering 
        \includegraphics[width=.4\linewidth]{logo-redis.png} 
        \caption{Redis} 
        \label{fig:redis} 
    \end{minipage}% 
    \begin{minipage}{.5\textwidth} 
        \centering 
        \includegraphics[width=.4\linewidth]{logo-celery.png} 
        \caption{Logo Celery} 
        \label{fig:celery} 
    \end{minipage}% 
\end{figure}

Infine RabbitMQ è un message-oriented middleware (detto anche broker di messaggistica) che implementa il protocollo Advanced Message Queuing Protocol (AMQP). Il server RabbitMQ è scritto in Erlang e si basa sul framework Open Telecom Platform[1] (OTP) per la gestione del clustering e del failover. Le librerie client per interfacciarsi a questo broker sono disponibili per diversi linguaggi.
Un broker di messaggi è un programma intermedio che traduce un messaggio dal protocollo di messaggistica formale del mittente al protocollo di messaggistica formale del ricevitore. I broker di messaggi sono elementi di telecomunicazione o reti di computer in cui le applicazioni software comunicano scambiando messaggi definiti in modo formale. 

\begin{figure}[!h] 
    \centering 
    \includegraphics[width=0.4\columnwidth]{logo-rabbitmq.png} 
    \caption{Logo RabbitMQ}
    label{fig:rabbitmq} 
\end{figure}

%**************************************************************
\section{Tecnologie utilizzate}

\subsection{Linguaggi}
Il principale linguaggio utilizzato è stato Python, in versione 3.7.3. Python è un linguaggio di programmazione ad alto livello, orientato agli oggetti adatto, tra gli altri usi, a sviluppare applicazioni distribuite, scripting, computazione numerica e system testing.  che si appoggia sull’omonima piattaforma software di esecuzione. Sebbene Python venga in genere considerato un linguaggio interpretato, in realtà il codice sorgente non viene convertito direttamente in linguaggio macchina. Infatti passa prima da una fase di pre-compilazione in bytecode, che viene quasi sempre riutilizzato dopo la prima esecuzione del programma, evitando così di reinterpretare ogni volta il sorgente e migliorando le prestazioni. La scelta della versione è ricaduta su quella ritenuta più stabile e che non creasse conflitti con gli altri strumenti e tecnologie.\newline{}
Per la gestione e configurazione dei file di impostazioni è stato utilizzato \textit{YAML}, un formato per la serializzazione di dati utilizzabile da esseri umani. Il nome definisce l'acronimo ricorsivo "YAML Ain't a Markup Language".

\begin{figure}[!h]
    \begin{minipage}{.5\textwidth} 
        \centering 
        \includegraphics[width=.4\linewidth]{logo-python.png} 
        \caption{Python} 
        \label{fig:python} 
    \end{minipage}% 
    \begin{minipage}{.5\textwidth} 
        \centering 
        \includegraphics[width=.4\linewidth]{logo-yaml.png} 
        \caption{Logo YAML} 
        \label{fig:yaml} 
    \end{minipage}% 
\end{figure}


\subsection{Framework}
Selenium e robe per test 

%**************************************************************