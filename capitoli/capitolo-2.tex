% !TEX encoding = UTF-8
% !TEX TS-program = pdflatex
% !TEX root = ../tesi.tex

%**************************************************************
\chapter{Processi e metodologie}
\label{cap:processi-metodologie}
%**************************************************************

\intro{In questo capitolo vengono elencati e descritti i casi d’uso emersi durante la fase di analisi del software precedente e del prodotto sviluppato, e mostrato, tramite opportune tabelle, il tracciamento dei requisiti.}\\

%**************************************************************
\section{Studio dei moduli presenti}

Prima di progettare e sviluppare il nuovo modulo di web crawling, ho effettuato un’analisi della struttura della piattaforma presente per comprenderne la struttura e le funzionalità offerte. Principalmente è stato fondamentale individuare l'organizzazione delle funzionalità di log, l'organizzazione dei file di configurazione e il codice principale da cui si avvia il programma. La struttura del modulo è risultata pulita, di facile comprensione e molto scalabile, richiedendo per l'inserimento del nuovo modulo l'aggiunta di poche e coincise linee di codice.

%**************************************************************
\section{Individuazione delle funzionalità}



\subsection{Funzionalità aggiuntive}

%**************************************************************
\section{Caratteristiche del modulo}
%**************************************************************