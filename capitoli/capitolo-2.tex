% !TEX encoding = UTF-8
% !TEX TS-program = pdflatex
% !TEX root = ../tesi.tex

%**************************************************************
\chapter{Analisi del problema}
\label{cap:analisi-del-problema}
%**************************************************************

\intro{In questo capitolo vengono elencati e descritti i casi d’uso emersi durante la fase di analisi del software precedente e del prodotto sviluppato, e mostrato, tramite opportune tabelle, il tracciamento dei requisiti.}\\

%**************************************************************
\section{Cos'è un web crawler?}

Un web crawler è un software che analizza i contenuti di una rete in un modo metodico e automatizzato. L'utilizzo più comune dei crawler avviene sul World Wide Web; tramite una lista di URL da visitare identifica tutti i collegamenti ipertestuali presenti nel documento e li aggiunge alla lista di URL da visitare. Data la natura della ricerca essa potrebbe protrarsi per un tempo indefinito, tuttavia, grazie ad alcune politiche si può influenzare l'ordine di ricerca, la frequenza e quali pagine evitare. I motori di ricerca odierni si basano su web crawler per poter ricercare ed indicizzare i vari siti internet.

\begin{figure}[!h] 
    \centering 
    \includegraphics[width=0.3\columnwidth]{chapter2-analysis/WebCrawlerArchitecture.png} 
    \caption{Architettura di un web crawler.}
\end{figure}

%**************************************************************
\section{Cosa significa web scraping?}

Con il termine web scraping ci si riferisce all'estrapolazione di dati dai siti web. Il termine generalmente comprende anche l'estrapolazione fatta manualmente tramite software anche se il termine principalmente si riferisce al processo automatizzato implementato tramite web crawler. L'attività principale di cui si occupa un web scraper è la trasformazione di informazioni libere in informazioni strutturate per renderle di più facile utilizzo per una successiva analisi. 

\begin{figure}[!h] 
    \centering 
    \includegraphics[width=0.3\columnwidth]{chapter2-analysis/webscraping.png} 
    \caption{Funzionamento di un web scraper.}
\end{figure}

%**************************************************************

\section{Studio dei moduli presenti}

Prima di progettare e sviluppare il nuovo modulo di web crawling, ho effettuato un’analisi della struttura della piattaforma presente per comprenderne la struttura e le funzionalità offerte. Principalmente è stato fondamentale individuare l'organizzazione delle funzionalità di log, l'organizzazione dei file di configurazione e il codice principale da cui si avvia il programma. La struttura del modulo è risultata pulita, di facile comprensione e molto scalabile, richiedendo per l'inserimento del nuovo modulo l'aggiunta di poche e coincise linee di codice.


\section{Individuazione degli obiettivi}

\subsection{Notazione obiettivi}
Si farà riferimento ai requisiti secondo le seguenti notazioni:
\begin{table}[!htbp]
  \centering
    \begin{tabularx}{\textwidth}{cX}
    \toprule
    \textbf{Notazione}	&	\textbf{Descrizione} \\
    \midrule
    OB			&	Requisiti obbligatori, vincolanti in quanto obiettivo primario richiesto dal committente. \\  
	\hline
    D			&	Requisiti desiderabili, non vincolanti o strettamente necessari, ma dal riconoscibile valore aggiunto. \\ 
	\hline
    OP			&	Requisiti opzionali, rappresentanti valore aggiunto non strettamente competitivo. \\
    \bottomrule
    \end{tabularx}%
  \label{tab:notazione-requisiti}%
  \caption{Notazione dei requisiti}
\end{table}%


\subsection{Tabella obiettivi}

\begin{table}[!htbp]
  \centering
    \begin{tabularx}{\textwidth}{cX}
    \toprule
    \textbf{Codice}	&	\textbf{Descrizione} \\
    \midrule
    OB1			&	Configurazione dell'ambiente di sviluppo. \\  
	\hline
    OB2			&	Scrittura del documento "Analisi dei Requisiti". \\ 
	\hline
    OB3			&	Scrittura della guida "README.md". \\
    \hline
    OB4			&	Progettazione del modulo di web crawling intelligente. \\
    OB5			&	Funzionamento su rete Tor. \\
    \hline
    OB6			&	Funzionamento utilizzando la libreria "requests". \\
    \hline
    OB7			&	Implementazione di "celery". \\
    \hline
    OB8			&	Configurazione dei file di log. \\
    \hline
    OB9			&	Ricerca url da analizzare tramite tag href. \\
    \hline
    OB10		&	Implementazione della ricerca di informazioni interessanti tramite regex. \\
    \hline
    OB11		&	Implementazione di priorità di analisi calcolate dinamicamente ed intelligentemente. \\
    \hline
    OB12		&	Implementazione di un sistema per riprovare url con i quali è fallita la connessione. \\
        
    \bottomrule
    \end{tabularx}%
  \label{tab:notazione-requisiti}%
  \caption{Notazione dei requisiti}
\end{table}%

\subsection{Funzionalità aggiuntive}

Oltre alle funzionalità sopra descritte, in seguito a vari confronti con il tutor aziendale, è stato deciso di arricchire il prodotto con delle nuove funzionalità. \newline{}
La prima riguarda l'estensione delle funzionalità pre esistenti su rete indicizzata, in questo modo si può avere un tracciamento più completo del flusso seguito dal modulo durante la ricerca delle informazioni rilevanti. Un'altra funzionalità che è stata ritenuta importante da aggiungere è stata la scrittura di informazioni di diagnostica su file csv facilmente utilizzabili sia dalla macchina che da persone. \newline{} Infine l'ultima funzionalità aggiunta è la realizzazione di una libreria per poter automaticamente creare grafici e rappresentare visivamente i dati raccolti.


%**************************************************************
\section{Caratteristiche del modulo}
%**************************************************************

Il modulo, oltre a fornire tutte le funzionalità descritte precedentemente, deve avere le
seguenti caratteristiche: deve essere scalabile e performante per permettere un aumento o diminuzione delle risorse; questo viene realizzato tramite una corretta progettazione mirata alla ottimizzazione dell'utilizzo di memoria per poter permettere a più istanze del programma di essere eseguite contemporaneamente tramite la libreria Celery.