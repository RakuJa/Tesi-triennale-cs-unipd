% !TEX encoding = UTF-8
% !TEX TS-program = pdflatex
% !TEX root = ../tesi.tex


%**************************************************************
\chapter{Conclusioni}
\label{cap:conclusioni}
%**************************************************************

%**************************************************************
\section{Raggiungimento degli obiettivi}
La maggior parte degli obiettivi è stata soddisfatta. Gli obiettivi non soddisfatti sono OP06 per scelta dell'azienda e D09 per mancanza di tempo. Gli obiettivi stesi, inizialmente, erano una serie di \emph{feature} desiderate, di cui non era stata fatta una analisi del tempo richiesto. Riuscire dunque a completare la maggioranza dei requisiti ha superato le aspettative iniziali dell'azienda.

%**************************************************************
\section{Conoscenze acquisite}
Questo stage ha contribuito a formarmi in un ambito a me completamente nuovo, quello del dark web e dei web crawler. Mi ha permesso di fare esperienza nel lavoro su strutture complesse, come quella della piattaforma di \emph{Cyber Threat Intelligence} proprietaria. Il progetto, in particolare, mi ha permesso di creare un nuovo modulo che coprisse una area prima non ricercata ed inoltre ha potuto aiutarmi nel rafforzare le mie conoscenze in aree più tecniche nell'ambito dei sistemi operativi, come l'utilizzo di sistemi unix, la configurazione di macchine e la configurazione di reti. A livello aziendale, invece, questo stage ha permesso di dotare la piattaforma aziendale dei mezzi necessari per poter effettuare scraping su buona parte del dark e clear web, indipendentemente dalla struttura del sito. Ciò significa che l’azienda attualmente possiede uno strumento personalizzato ed estendibile per soddisfare le proprie esigenze. \newline{}
Nel mio percorso di stage ho avuto modo di conoscere nuove tecnologie e strumenti, imparando ad apprezzare pregi e difetti di ognuno, per questo è stato fondamentale apprendere rapidamente tecnologie mai utilizzate in precedenza.
Da questa esperienza ho potuto approfondire la mia conoscenza del linguaggio \emph{Python} imparato durante il corso di Ingegneria del Software e le conoscenze di sicurezza informatica imparate nel corso di Cybersecurity, applicandolo allo sviluppo di un applicativo scalabile e performante.
\newpage
%**************************************************************
\section{Valutazione personale}

Considero l'esperienza di stage nel complesso positiva. Quasi tutti gli obiettivi sono stati portati a termine ed il modulo realizzato è funzionante ed operativo oltre che facilmente estensibile. In azienda mi è stata data la possibilità di svolgere il tirocinio in modalità duale, scegliendo autonomamente i giorni in cui lavorare in presenza e quelli in cui lavorare da remoto. Il progetto è stato seguito dal tutor aziendale Matteo Neri, che si è sempre dimostrato aperto al confronto e a lasciarmi autonomia decisionale quando possibile. \newline{}
Fin dal primo giorno di sviluppo ero sicuro che sarei riuscito a completare il progetto in maniera soddisfacente nei tempi richiesti e così è stato, imprevisti e problematiche non sono mancati ma sono sempre stati risolti.
