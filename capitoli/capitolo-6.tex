% !TEX encoding = UTF-8
% !TEX TS-program = pdflatex
% !TEX root = ../tesi.tex


%**************************************************************
\chapter{Conclusioni}
\label{cap:conclusioni}
%**************************************************************

%**************************************************************
\section{Raggiungimento degli obiettivi}
La maggior parte degli obiettivi è stata soddisfatta. Gli obiettivi non soddisfatti sono OP06 per scelta dell'azienda e D09 per mancanza di tempo, gli obiettivi stesi inizialmente infatti erano una serie di "feature" desiderate di cui non era stata fatta una analisi del tempo richiesto. Riuscire dunque a completare la maggioranza dei requisiti è stato qualcosa che ha superato le aspettative iniziali dell'azienda.

%**************************************************************
\section{Conoscenze acquisite}
Nel mio percorso di stage ho avuto modo di conoscere nuove tecnologie e strumenti, imparando ad apprezzare pregi e difetti di ognuno, per questo è stato fondamentale apprendere rapidamente tecnologie mai utilizzate in precedenza.
Da questa esperienza ho potuto approfondire la mia conoscenza del linguaggio \textit{Python} imparato durante il corso di Cybersecurity e Ingegneria del Software, applicandolo allo sviluppo di un applicativo scalabile e performante.

%**************************************************************
\section{Valutazione personale}

Considero l'esperienza di stage nel complesso positiva. Quasi tutti gli obiettivi sono stati portati a termine ed il modulo realizzato è funzionante ed operativo oltre che facilmente estensibile. In azienda mi è stata data la possibilità di svolgere il tirocinio in modalità duale, scegliendo autonomamente i giorni in cui lavorare in presenza e quelli in cui lavorare da remoto. Il progetto è stato seguito dal tutor aziendale Matteo Neri, che si è sempre dimostrato aperto al confronto e a lasciarmi autonomia decisionale quando possibile. \newline{}
Fin dal primo giorno di sviluppo ero sicuro che sarei riuscito a completare il progetto in maniera soddisfacente nei tempi richiesti e così è stato, imprevisti e difficoltà non sono mancati ma sempre legati a errori di celery che sono in attesa di fix (come memory leak).
